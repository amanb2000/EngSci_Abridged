\documentclass[a4paper,12pt]{report}
\usepackage{color}
\usepackage{graphicx}
\usepackage{subfig}
\usepackage{listings}
\usepackage{media9}
\usepackage{hyperref}
\usepackage{amssymb}

% \def\reals{{\rm I\!R}}
\def\reals{\mathbb{R}}
\def\integers{\mathbb{Z}}

\definecolor{dkgreen}{rgb}{0,0.6,0}
\definecolor{gray}{rgb}{0.5,0.5,0.5}
\definecolor{mauve}{rgb}{0.58,0,0.82}

\begin{document}

\title{ECE355: Signal Analysis and Communications}
\author{Aman Bhargava}
\date{September-December 2020}
\maketitle

\tableofcontents

\section{Introduction and Course Information}

This document offers an overview of the ECE355 course. They comprise my condensed course notes for the course. No promises are made relating to the correctness or completeness of the course notes. These notes are meant to highlight difficult concepts and explain them simply, not to comprehensively review the entire course.

Primary course topics include:

\begin{enumerate}
\item Signals and Systems (Chapter 2).
\item Frequency Domain Analysis (Chapters 3-5).
\item Sampling (Chapter 9).
\item Introduction to Communication Systems (Chapter 8).
\end{enumerate}

\paragraph{Course Information}
\begin{itemize}
\item Professor: Ben Liang
\item Course: Engineering Science, Machine Intelligence Option
\item Term: 2020 Fall
\end{itemize}




\chapter{Signal Basics}

\section{Definitions}

\paragraph{Two types of signals: } Continuous ($f(x)$ defined $\forall x \in \reals$) and Discrete $f(n)$ defined $\forall n \in \integers$.

\paragraph{Power and Energy of a signal: }
\begin{itemize}
	\item Power of $x(t)$ is $|x(t)|^2$.
	\item Energy of $x(t)$ is defined on interval $[t_1, t_2]$ as 
	$$E_{[t_1, t_2]} = \int_{t_1}^{t_2} |x(t)|^2 dt$$
	$$E_{n_1 \leq n \leq n_2} = \sum_{n=n_1}^{n_2} |x[n]|^2$$

	\item Average power of in $[t_1, t_2]$:
	$$P_{[t_1, t_2]} = \frac{E_{[t_1, t_2]}}{t_2 - t_1}$$
	$$P_{[n_1, n_2]} = \frac{E_{n_1 \leq n \leq n_2}}{n_2 - n_1 + 1}$$
	\item Total Energy: 
	$$E_{\infty} = \lim_{T\to\infty} \int_{-T}^T |x(t)|^2 dt$$
	$$E_{\infty} = \lim_{N\to\infty} \sum_{n = -N}^{N} |x[n]|^2$$
\end{itemize}


\section{Signal Transformations}

\paragraph{Time Shifting: } \textit{Shifts $t_0$ units RIGHT}
$$y(t) = x(t-t_0)$$
$$y[n] = x[n-n_0]$$

\paragraph{Time Scaling: } \textit{Speeds original signal up by factor $a$)} (or slowed down by factor $\frac{1}{a}$). Time reversal occurs when $a<0$.
$$y(t) = x(at)$$

\paragraph{Continuous Scaling AND Shifting: } It is important to remember the following steps for $y(t) = x(at+b)$

\begin{enumerate}
\item \textbf{SHIFT}: $v(t) = x(t+b)$.
\item \textbf{SCALE}: $y(t) = v(at)$.
\end{enumerate}

\paragraph{Discrete Time Scaling AND Shifting: } Remember to IGNORE fractional indexes. Interpolation for `slowing down' a signal is a poorly defined process that will be covered later.

\section{Periodic Signals}

\paragraph{Definition: } A signal is periodic iff $\exists T > 0$ s.t. $x(t+T) = x(t)$ $\forall t\in\reals$. 

\begin{itemize}
\item $T$ is the period of the signal.
\item \textbf{Fundamental} period is the smallest possible $T$.
\item If $x(t)$ is constant, then the fundamental period is undefined.
\end{itemize}

\section{Even and Odd Signals}

\paragraph{Even: } $x(t) = x(-t)$
\paragraph{Odd: } $x(-t) = -x(t)$

\textbf{ANY SIGNAL} can be decomposed into an even and odd component.

$$x_{even}(t) = \frac{1}{2} (x(t) + x(-t))$$
$$x-{odd}(t) = \frac{1}{2} (x(t) - x(-t))$$

$$x(t) = x_{even}(t) + x_{odd}(t)$$


\section{Complex Exponential}
\paragraph{Function Family: } $x(t) = c e^{at}$, $c, a\in \mathbb{C}$

\subsection{}











\end{document}

